\chapter{ORM}

\section{リレーション}

\subsection{多対多}

関連テーブルを作ります。例として、GroupテーブルとLocationテーブルを相互に参照する関係を記述します。

\begin{lstlisting}[caption=多対多]
GroupLocation = Table('group_location', Base.metadata,
  Column('group_id', ForeignKey('groups.id'), primary_key=True),
  Column('location_id', ForeignKey('locations.id'), primary_key=True)
)
\end{lstlisting}

各テーブル間のリレーションは \textbf{relation}パラメータを与えることで実現します。
また、relationship.secondary引数にこの関連テーブルを指定します。



\begin{lstlisting}[caption=テーブル定義]
class Group(Base):
  __tablename__ = "groups"
  id = Column(Integer, primary_key=True)
  name = Column(String(255))
  label = Column(String(255))
  users = relationship("UserTable", back_populates="group")
  locations =  relationship(
      "Location",
      secondary=GroupLocation,
      backref="groups")

class Location(Base):
  __tablename__ = "locations"
  id = Column(Integer, primary_key=True)
  name = Column(String(255))
  label = Column(String(255))
  latitude = Column(Float)
  longitude = Column(Float)
\end{lstlisting}
