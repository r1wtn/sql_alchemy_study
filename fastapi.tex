\chapter{FastAPI}

本章では、WebフレームワークであるFastAPIを利用し、SQLAlchemyの実用例を見ていきます。

\section{用語}

\begin{dfn}
  \textbf{パス}
\end{dfn}
最初の / から始まるURLの最後の部分です。

\begin{dfn}
  \label{operation}
  \textbf{オペレーション}
\end{dfn}
HTTPメソッドの一つを指します。

\begin{dfn}
  \textbf{パスオペレーション関数}
\end{dfn}
特定のパスに対して要求があったときに実行される関数です。

\begin{dfn}
  \textbf{モデル}
\end{dfn}
ここでは、データベースのテーブルをSQLAlchemyで実装したオブジェクトを指します。

\begin{dfn}
  \textbf{スキーマ}
\end{dfn}
ここでは、Pydanticによって定義されたオブジェクトのことを指します。



\section{Sessionの注入}
FastAPIの基本概念であるDendency Injectionを利用して、各パスオペレーション関数にてデータベースを操作するためのSessionを使えるようにします。
\ref{session.interface}にて述べたように、Sessionはデータベース操作のたびに生成され、処理が終わったら破棄されます。
この部分を実現するジェネレータ関数を、以下のように定義します。

\begin{lstlisting}[caption=セッションジェネレータ][H]
# sessionmaker to make Session with api version 2.0 (future)
Session = sessionmaker(engine, future=True)

# dependency for each api call
def get_db_session():
    db = Session()
    try:
        yield db
    finally:
        db.close()
\end{lstlisting}

\begin{lstlisting}[caption=パスオペレーション関数][H]
  @router.get('/get', response_model=List[schema.Group])
  async def get_groups(request: Request, session: Session = Depends(get_db_session)):
      stmt = select(Group)
      groups = session.execute(stmt).scalars().all()
      return groups
\end{lstlisting}

Dependsを利用してget\_db\_session関数をパスオペレーションにて呼び出すと、
呼ばれたときにSessionがインスタンス化されて利用できるようになります。
そしてレスポンスが返されてsession変数がスコープを抜けると、finallyに定義されたcloseが呼ばれ、sessionが必ず破棄されます。


\section{スキーマの定義}
データベース上の各モデルに対応するスキーマを定義する必要があります。
スキーマを利用することで、SQLAlchemyのモデルを自動的にMap Objectに変換してレスポンスを返せるようになります。
例えば、以下のようにLocationというモデルがデータベースに定義されているとします。

\begin{lstlisting}[caption=Location Model]
class Location(Base):
  __tablename__ = "locations"
  id = Column(Integer, primary_key=True)
  name = Column(String(255))
  label = Column(String(255))
  latitude = Column(Float)
  longitude = Column(Float)
\end{lstlisting}

これに対応するスキーマは以下のようにかけます。


\begin{lstlisting}[caption=Location schema]
class LocationBase(BaseModel):
  name: str
  label: str
  latitude: Optional[float]
  longitude: Optional[float]

class LocationCreate(LocationBase):
  pass

class Location(LocationBase):
  id: str
  class Config:
      orm_mode = True
\end{lstlisting}

ここで、LocationCreateとLocationを分けているのは、\textbf{リクエストボディで利用するのか}、\textbf{レスポンスボディで利用するのか}によって使い分けるためです。
まず、リクエストボディとしてLocationのMap Objectが送られてくるようなケースを想定します。
パスオペレーション関数は次のようになるでしょう。

\begin{lstlisting}[caption=Create New Location]
@router.post('/post')
async def post_location(
    location: schema.LocationCreate,
    session: Session = Depends(get_db_session)
):
    ...
    return location
\end{lstlisting}


リクエストボディにschema.LocationCreateを定義しています。
もしこれをschema.Locationに変更するとどうなるでしょうか。
Locationは、必須フィールドであるidを持っています。
これはモデルのprimary keyとして設定しているidに対応させるためです。
通常、idは自動インクリメントで振られるため、ユーザからのリクエストボディに含まれることはありません。
しかし、これがリクエストボディのスキーマとして与えられてしまうと、idがないためにバリデーションエラーになり、リクエストを受け付けることができません。
そのため、リクエストボディには専用のLocationCreateスキーマを定義して利用しています。
ではGETの方はどうでしょうか。

\begin{lstlisting}[caption=Get Location List]
@router.get('/get', response_model=List[schema.Location])
async def get_location(request: Request, session: Session = Depends(get_db_session)):
    stmt = select(Location)
    locations = session.execute(stmt).scalars().all()
    return locations
\end{lstlisting}


こちらのURLにアクセスすると、Locationテーブルのデータの配列を全て取得することができます。
ポイントは、\textbf{response\_model}にLocationスキーマを与えている点です。
4行目のlocationsは、クエリを発行して得られたlocationモデルの配列です。
response\_modelを与えることにより、このモデルを自動的にMap objectに変換して返してくれます。
ポイントは、モデルとスキーマのフィールド定義が完全に一致していることです。
フィールドの型が異なったり不足するフィールドがあると、バリデーションエラーにより値を返すことができません。
ちなみに、response\_modelを定義せずにMap Objectを返そうとすると、以下のような感じになります。

\begin{lstlisting}[caption=スキーマを使わないGET]
@router.get('/get')
async def get_location(request: Request, session: Session = Depends(get_db_session)):
    stmt = select(Location)
    locations = session.execute(stmt).scalars().all()
    data = []
    for l in locations:
        data.append({
            "id": l.id,
            "label": l.label,
            "name": l.name,
            "latitude": l.latitude,
            "longitude": l.longitude
        })
    return data
\end{lstlisting}

モデルから必要なフィールドを取り出して、Map Objectに変換したものを配列に詰めて返します。


\section{各オペレーションの基本構文}
この節では、各オペレーション(定義\ref{operation})における基本的な構文を見ていきます。

\subsection{GET}

\subsubsection*{全てのオブジェクトを取り出す}

\begin{lstlisting}[caption=Get Location List]
  @router.get('/get', response_model=List[schema.Location])
  async def get_location(session: Session = Depends(get_db_session)):
      stmt = select(Location)
      locations = session.execute(stmt).scalars().all()
      return locations
  \end{lstlisting}
  

\subsubsection*{条件に合致するオブジェクトを取り出す}

\begin{lstlisting}[caption=Get Location on condition]
  @router.get('/get', response_model=schema.Location)
  async def get_location(location_id: str, session: Session = Depends(get_db_session)):
      stmt = select(Location).filter_by(id=location_id)
      location = session.execute(stmt).scalar_one()
      return location
\end{lstlisting}

\subsubsection*{リレーションがある場合}
GroupとLocationが多対多のリレーションを貼っているとします。
すると、モデル定義は以下のようになります。
\begin{lstlisting}[caption=モデル定義]
GroupLocation = Table('group_location', Base.metadata,
  Column('group_id', ForeignKey('groups.id'), primary_key=True),
  Column('location_id', ForeignKey('locations.id'), primary_key=True)
)

class Group(Base):
  __tablename__ = "groups"
  id = Column(Integer, primary_key=True)
  name = Column(String(255))
  label = Column(String(255))
  users = relationship("UserTable", back_populates="group")
  locations =  relationship(
      "Location",
      secondary=GroupLocation,
      backref="groups")

class Location(Base):
  __tablename__ = "locations"
  id = Column(Integer, primary_key=True)
  name = Column(String(255))
  label = Column(String(255))
  latitude = Column(Float)
  longitude = Column(Float)
\end{lstlisting}

GroupをGETで取り出すと、以下のようになります。




\section{プロジェクトの構成}
参考までに、FastAPIxSQLAlchemyのプロジェクト構成を載せておきます。

\begin{lstlisting}[caption=プロジェクト構成]
models/
  - __init__.py    # ここにmodelの定義をここにします。
schemas/
  - __init__.py    # スキーマの定語をここにします。
routers/
  - __init__.py    # APIRouterの定義をここにします。
  - some_router.py
config.py    # 各種設定値を格納します。
main.py    # エントリポイント
\end{lstlisting}
